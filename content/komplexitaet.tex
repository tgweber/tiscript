\chapter{Vom Reisen}\label{complex}
Recap: Anzahl Bänder sub-exponentiell auf Einbandmaschine simulierbar.
\section{Effizienz: P}
\begin{itemize}
    \item ``Effiziente'' Algorithmen $\rightarrow$ TM hat maximal polynomielle Laufzeit.
    \item ``Effiziente'' Probleme: $\exists TM$ mit maximal polynomieller Laufzeit,
        die die formale sprache erkennt 
    \item Polynomielle Laufzeit: $t_{tm} \in O(n^k)$ für ein $k \in \mathbb{N}$.
    \item Definition P = Alle Sprachen,
        die von einer (D)TM mit maximal polynomieller Laufzeit erkannt werden:
        $P = \{L| \exists tm (tm(L) = L \wedge t_{tm} \in O(n^k) \wedge k \in \mathbb{N})\}$
\end{itemize}
\section{Determinismus und Nicht-Determinismus}
\subsection{Nichtdeterministische Finite Automaten}
Ein Nichtdeterministischer Finiter Automat (NFA) ist ein 5-Tupel:
$A = [Z, \Sigma, \delta, S, E]$:
\begin{itemize}
    \item Z ist die Menge der Zustände.
    \item $\Sigma$ ist das Eingabealphabet.
    \item $\delta: Z \times \Sigma \rightarrow \mathcal{P}(Z)$ ist die Übergangsfunktion.
    \item $S \subseteq Z$ ist die Menge der Startzustände.
    \item $E \subseteq Z$ ist die Menge der Endzustände.
\end{itemize}

Ein Schnappschuss eines NFAs ist analog zum Schnappschuss eines DFAs definiert.

Ein Schnappschussbaum ist eine sich verzweigende Liste von Schnappschüssen,
wobei die Verzweigung aus $\delta$ hervorgeht.

Die von einem NFA akzeptierte Sprache L ist definiert als alle Wörter,
für die es einen Pfad im Schnappschuss-Baum gibt,
der an einem Blatt $[\epsilon, z_e]$ endet mit $z_e \in E$.

\subsubsection{Potenzmengenkonstruktion}
Wir können einen DFA $A_d = [Z, \Sigma, \delta_d, z_i, E]$
leicht in einen NFA $A_n [Z, \Sigma, \delta_n, S, E]$ umwandeln
($\Sigma$, $Z$ und $E$ bleiben identisch):
\begin{itemize}
    \item $\delta_n(z, a) = \{\delta_d(z, a)\}$
    \item $S = \{z_i\}$
\end{itemize}

Wir können mit folgender Potenzmengenkonstruktion
einen NFA $A_n =  [Z_n, \Sigma, \delta_n, S, E_n]$ umwandeln
einen DFA $A_d = [Z_d, \Sigma, \delta_d, z_i, E_d]$
umwandeln
($\Sigma$ bleibt identisch):
\begin{itemize}
    \item $Z_d = \mathcal{P}(Z_n)$
    \item $\delta_d(z, a) = \bigcup\limits_{z \in Z'} \delta_n(z, a)$
    \item $z_i = S$
    \item $E_d = \{z \in Z_d| z \cap E_n \neq \emptyset \}$
\end{itemize}

Meist hat ein DFA,
der aus dieser Konstruktion gewonnen hat überflüssige Zustände,
daher führen wir die folgende Abkürzung ein:
\begin{enumerate}
    \item Zeichne Startzustand $z_i$ (beinhaltet alle Startzustände $S$).
    \item Wiederhole für jedes $a \in \Sigma$ und jeden bereits gezeichneten Zustand $z_d$,
        der noch keinen Folgezustand für $a$ hat:
        \begin{enumerate}
            \item Identifiziere alle Folgezustände auf dem NFA für a:\\
                $z_d' = \{z_n' \in Z_n| z = \delta_n(z_n,a) \wedge z_n \in z_d\}$\\
            \item Zeichne $z_d'$, falls dieser noch nicht existiert.
            \item Zeichne den Pfeil $z_d \xrightarrow[]{\text{a}} z_d'$.
        \end{enumerate}
    \item Falls es noch Zustände gibt, für die es für ein $a \in \Sigma$ keinen Folgezustand gibt,
        gehe zu 2.
\end{enumerate}


\subsection{Nichdeterministische Turingmaschinen}

Eine nicht-deterministische Turingmaschine NTM ist ein 7-Tupel 
$TM = [Z, \Sigma, \Gamma, \delta, z_0, \square, E]$:
\begin{itemize}
    \item $Z$ ist die Menge der Zustände.
    \item $\Sigma$ ist das Eingabealphabet.
    \item $\Gamma$ ist das Arbeitsalphabet. Es gilt: $\Sigma \subsetneq \Gamma$.
    \item $\delta: Z \times \Gamma \rightarrow \mathcal{P}(\Gamma \times Z \times \{R,L,-\})$
        ist die \emph{partielle} Übergangsfunktion.
    \item $z_0$ ist der Startzustand.
    \item $\square$ ist das Blank.
        Es gilt $\square \in \Gamma \wedge \square \notin \Sigma$.
    \item E ist die Menge der Endzustände.
\end{itemize}

Wie bei NFAs, ergibt sich aus einem Inputwort ein Schnappschussbaum,
die akzeptierte Sprache lässt sich äquivalent definieren.

\section{NP}
NP ist die Menge aller Sprachen,
die von einer NTM mit maximal polynomieller Laufzeit erkannt werden:
\[
    NP = \{L| \exists ntm (ntm(L) = L \wedge t_{ntm} \in O(n^k) \wedge k \in \mathbb{N})\}
\]
\subsection{Komplexitätstheoretische Reduktionen}

Eine Sprache A ist auf B polynomiell reduzierbar ($A \leq_p B$), 
wenn es eine totale Funktion $f$ mit $f \in O(n^k) (k \in \mathbb{N})$ gibt,
so dass gilt: $w \in A \leftrightarrow f(w) \in B$.

%TODO: Ausgedrückt mit einer Turingmaschine.

\subsection{NP-Schwere und NP-Vollständigkeit}

Das Problem, das duch eine formale Sprache $P$ repräsentiert wird, ist NP-schwer
genau dann, wenn für alle $L \in NP$ gilt $L \leq_p P$

Das Problem, das durch eine formale Sprache $P$ repräsentiert wird, ist NP-vollständig
genau dann, wenn $P$ NP-schwer ist und $P \in NP$.

\begin{itemize}
    \item SAT
    \item Reduktionsbaum
\end{itemize}

\section{P = NP?}

Wenn A NP-vollständig ist und wir zeigen können, dass $A \in P$, dann gilt $P = NP$:
Es reicht zu zeigen, dass ein NP-vollständiges Problem in P liegt,
dann folgt: für alle Probleme in NP gibt es eine deterministische Turingmaschine dtm,
welche die formale Sprache des Problems erkennt und für die gilt $t_dtm \in O(n^k), k \in \mathbb{N}$.

%Annahme $P \neq NP$.
%Konsequenzen wenn identisch.

\subsection*{Aufgaben}
\begin{enumerate}
    \item Sei der NFA wie in \autoref{fig:nfaex} gegeben.
        Determinisieren sie den NFA mit der Potenzmengenkonstruktion (nicht erreichbare Zustände dürfen ausgelassen werden).
\end{enumerate}

\begin{figure}[ht] % ’ht’ tells LaTeX to place the figure ’here’ or at the top of the page
\centering % centers the figure
\begin{tikzpicture}
	\node[state, initial] (zi) {$z_i$};
	\node[state, right of=zi] (z1) {$z_1$};
    \node[state, below of=zi] (z2) {$z_2$};
    \node[state, accepting, right of=z2] (z3) {$z_3$};
	\draw 
        (zi) edge[above] node{0} (z1)
        (zi) edge[below left] node{0} (z2)
        (zi) edge[below] node{1} (z3)
        
        (z1) edge[above right] node{1} (z3)
        (z1) edge[loop above] node{1} (z1)
        
        (z2) edge[above] node{0} (z3)
        (z2) edge[above] node{1} (z1)
        (z2) edge[loop below] node{0} (z2)
    ;
\end{tikzpicture}
\caption{Zu determinisierender NFA}\label{fig:nfaex}
\end{figure}
