\chapter{Das Hello World der Automatentheorie}
In diesem Kapitel werden wir das \textbf{MATCH}-Problem genauer unter die Lupe nehmen
und dabei unseren ersten Automaten-Typ formal einführen:
\begin{center}
``Als Einkaufsmanager:in möchte ich, dass nur valide Lieferanten-IDs gespeichert werden
(beginnend mit einem ``L'', gefolgt von einer Zahl),
um Fehler in der Datenbank zu vermeiden.''
\end{center}
Abstrahieren und Formalisieren wir das Problem bekommen wir folgende informelle Formulierung:

\begin{itemize}
\item \textbf{Gegeben}: Eine Zeichenfolge (String)
\item \textbf{Gesucht}: Entspricht die Zeichenfolge einem bestimmten Muster?
\end{itemize}
Die 
\section{Deterministisch Finite Automaten}
\section{Reguläre Ausdrücke}
\section{Minimalisierung}
\section{Grenzen endlicher Automaten}
