\chapter{Einleitung}
\section{Warum theoretische Informatik?}
Warum lohnt sich die Beschäftigung mit theoretischer Informatik?
Die kurze Antwort: Sie liefert einen Teil des nötigen Handwerkszeug,
um die Potentiale und Grenzen der technischen Informationsverarbeitung zu umreißen.
Die lange Antwort soll dieses Skript geben.

Das Skript wurde begleitend zur Veranstaltung ''Theoretische Informatik für Wirtschaftsinformatiker''
an der DHBW Heilbronn erstellt.
Wir folgen in diesem Skript einem kartesischen Ansatz:
Wir gehen vom maximalen Zweifel aus und setzen die theoretische Informatik 
als potentiellen Nichtsnutz auf die Anklagebank.
Schritt für Schritt werden wir die Jury (die Studierenden) von der Unschuld dieser
Disiplin überzeugen.
Wir beginnen mit der Feststellung, dass uns als Informatiker:innen die Nützlichkeit
eines Werkzeugs dann einleuchtet, wenn es hilft Probleme zu lösen. 
Daher ist der Problembegriff im Mittelpunkt dieses Skripts.

Was also ist ein Problem?
Bevor wir diese Frage grundlegend beantworten können,
ist es hilfreich ein paar Probleme kennen zu lernen,
wie sie im IT-Alltag auftreten können.

\section{Probleme}
Wir werden nun ein paar alltägliche Probleme als Userstories wiedergeben.
Eine Userstory wird in der agilen Softwareentwicklung genutzt um Anforderungen zu spezfizieren.
Es folgt typischerweise diesem Format:
''Als \textless Rolle\textgreater\ möchte ich \textless Feature\textgreater,
um \textless Ziel\textgreater\ zu erreichen.''

\begin{itemize}
    \item \textbf{MAX}: Als Filialleiterin möchte ich das Produkt kennen,
        dass sich am besten verkauft, um meine Lieferungen zu optimieren.
    \item \textbf{MATCH}: Als Einkaufsmanager möchte ich,
        dass nur valide Lieferanten-IDs gespeichert werden,
        um Fehler in der Datenbank zu vermeiden.
    \item \textbf{HALT}: Als Product Owner des Serverless-Computings möchte ich,
        dass nur terminierende Code-Snippets deployt werden,
        um die verfügbaren Ressourcen effizient zu nutzen.
    \item \textbf{ROUTE}: Als Kommissioniererin möchte ich,
        dass morgens die optimalen Routen für die LKWs in die 24 Filialen berechnet werden,
        um die Lieferzeit zu minimieren.
    \item \textbf{QUANTUMSEC}: Als Chief Security Officer möchte ich,
        dass die firmen-interne Kommunikation quantensicher verschlüsselt wird,
        um die Vertraulichkeit der Absprachen zu garantieren.
\end{itemize}

Alle Probleme treten in einer Form auf,
wie sie viele IT-Angestellten täglich vorfinden, dann analysieren und schließlich lösen müssen.
Die theoretische Informatik kann auch einiges über diese Probleme sagen:
\begin{itemize}
    \item Mindestens ein Problem ist sehr schwer zu lösen.
    \item Mindestens ein Problem ist sehr einfach zu lösen.
    \item Mindestens ein Problem ist nicht lösbar. 
    \item Bei mindestens einem Problem ist unklar, ob es sich lösen lässt.
\end{itemize}

Am Ende dieses Skriptes soll klar sein, welches dieser Probleme in welche dieser Kategorien fällt.
An diesem Punkt im Text wollen wir uns aber zunächst damit begnügen,
dass ein Fach Nutzen bringt, dass es erlaubt, diese Probleme auf einen Blick als leicht oder
schwer, unlösbar oder von unbekannter Komplexität zu kategorisieren. 

\section{Abstraktion}

Um die Probleme aus dem vorigen Absatz für die Bearbeitung mit den Werkzeugen der
theoretischen Informatik vorzubereiten ist noch etwas an Arbeit notwendig.
Ein grundlegendes Instrument nicht nur der theoretischen Informatik, sondern der Informatik
überhaupt ist die Abstraktion, also die ausschließliche Betrachtung der Aspekte eines Sachverhalts,
die für die Lösung eines Problems wesentlich sind.

So ist z.B. auf den ersten Blick unwesentlich, ob im \textbf{ROUTE}-Problem 2, 24 oder 100 Filialen angefahren werden,
ob es überhaupt Filialen sind (und nicht Lager, Tankstellen, oder Adressen von Kund:innen),
oder ob die Strecke morgens oder abends,
mit LKWs oder Kleintransportern befahren werden.
Viele dieser Details scheinen für die algorithmische Behandlung des Problems irrelevant.

Der Kern des Problems liegt in einer abstrakten Datenstruktur:
ein Graph dessen Knoten mit einer Kostenfunktion verbunden sind:
Die Knoten des Graphen sind die Filialen (oder Lager, ...),
die Kostenfunktion auf den Kanten ist die Dauer der Fahrt zwischen den Filialen (oder der Benzinverbrauch, die Distanz, ...).
Wir kommen mit Hilfe der Abstraktion also auf eine neue Formulierung des Problems:
\begin{itemize}
    \item \textbf{Gegeben:} Ein Graph mit einer Kostenfunktion
    \item \textbf{Gesucht:} Die Folge von einer Teilmenge der Knoten, für die die Kostenfunktion minimal ist.
\end{itemize}

Der Vorgang der Abstraktion ist leider weder eindeutig noch trivial.
Es ist zum Beispiel durchaus möglich,
dass der Zeitpunkt der Fahrten eben \emph{nicht} irrelevant ist,
da die gewählte Kostenfunktion (Dauer der Fahrt)
stark von der Verkehrssituation abhängt,
die wiederum abhängig von der Tageszeit ist.
Es gibt daher nicht einen eindeutig ''richtigen'' Weg von einer Userstory zu einem abstrakten Problem zu kommen,
wie es mit den Mitteln der theoretischen Informatik behandelt werden kann.
Aber jede der möglichen Abstraktionen wird letztlich in dieser Form resultieren:
\begin{itemize}
    \item \textbf{Gegeben:} Eine oder mehrere Datenstruktur(en)
    \item \textbf{Gesucht:} Ein (potentiell komplexer) Ausgabewert
\end{itemize}
Diese Form der Problemangabe wollen wir ''informell'' nennen.

Wir halten fest, dass die Abstraktion ein Werkzeug ist,
Probleme auf ihren Kern zu reduzieren.
Diese abstrakte, informelle Form des Problems hat einen Umfang, der klein wie möglich und so groß wie nötig ist.

Allerdings haben wir das \textbf{ROUTE}-Problem noch nicht in der Form angegeben,
die nötig ist, um es mit den Werkzeugen der theoretischen Informatik zu bearbeiten.
Ganz zu schweigen von den anderen Userstories.
Software-Entwickler:innen wären mit unserem Zwischenergebnis vielleicht schon zufrieden,
denn sie wüssten, wie man ein informelles Problem in die Form ihrer Programmiersprache bringen kann.
Welche Form die theoretische Informatik verlangt, soll im nächsten Abschnitt diskutiert werden.

\section{Formalisierung}

Die Informatik als Wissenschaft der Informationsverarbeitung bedingt,
dass wir die Lösung von Problemen als Anweisung zur Manipulation von Zeichen verstehen,
die von Maschinen ausgeführt werden:
\begin{itemize}
    \item Wir kodieren Bilder, Musik und Texte als binäre Zeichenfolgen
    \item Wir kodieren Anweisungen, was mit Bildern, Musik und Texten passieren soll, ebenfalls als binäre Zeichenfolgen
\end{itemize}

TBC.

\section{Weitere Werkzeuge der theoretischen Informatik}

\section{Drei grundlegende Fragen}
Zusammen mit dem Begriff eines Problems und seiner Formalisierung ist also der
\emph{Berechenbarkeitsbegriff} im Herzen dieses Skriptes.
Wir hätten auch andere Begriff auswählen können, um in die Ideen der theoretischen Informatik einzuführen:
Formale Sprachen, Automaten, Information, Aufwand, Überprüfbarkeit wären solche Kandidaten gewesen.
Das Problem und die Berechenbarkeit erlauben uns aber eine ganz Praxis-nahe Einführung.
Die anderen Begriffe werden wir in diesem Skript streifen, aber nicht in den Fokus setzen.

Mit unseren beiden Kernbegriffen können wir die drei Grundfragen dieses Skriptes stellen:
\begin{enumerate}
    \item Was lässt sich effizient berechnen?
    \item Was lässt sich überhaupt berechnen?
    \item Welche Rolle spielt die Formalisierung der Berechenbarkeit für die zwei ersten Fragen?
\end{enumerate}

Diese drei Fragen korrespondieren zu drei Grunddisziplinen der theoretischen Informatik:
\begin{enumerate}
    \item Komplexitätstheorie
    \item Berechenbarkeitstheorie
    \item Theorie der formalen Sprachen und der Automaten
\end{enumerate}

Im Rest des Skriptes werden wir die drei Fragen bearbeiten und
dabei die drei Grunddsiziplinen und ihre Fragestellungen, Ergebnisse und Methodiken kennenlernen.
