\chapter{Einleitung}
\section{Warum theoretische Informatik?}
Warum sollte man sich mit theoretischer Informatik beschäftigen?
Die kurze Antwort: Sie liefert einen Teil des nötigen Handwerkszeug,
um die Potentiale und Grenzen der technischen Informationsverarbeitung zu umreißen.
Die lange Antwort soll dieses Skript geben.
Es wurde begleitend zum Wirtschaftsinformatikstudium an der DHBW Heilbronn erstellt
und folgt einem kartesischen Ansatz:
Wir gehen vom maximalen Zweifel aus und setzen die theoretische Informatik 
als potentiellen Nichtsnutz auf die Anklagebank.
Schritt für Schritt werden wir die Jury (die Studierenden) von der Unschuld dieser
Disiplin überzeugen.
Wir beginnen mit der Feststellung, dass uns als Informatiker:innen die Nützlichkeit
eines Werkzeugs dann einleuchtet, wenn es hilft Probleme zu lösen. 
Daher ist der Problembegriff im Mittelpunkt dieses Skripts.

Was also ist ein Problem?
\section{Probleme}
\section{Abstraktion}
\section{Formalisierung}
\section{Weitere Werkzeuge der theoretischen Informatik}
\section{Drei grundlegende Fragen}
