\chapter{Vorwort}

Warum braucht es eine weitere (deutschsprachige) Einführung
in die theoretische Informatik?
Es gibt ja bereits Bücherregale gefüllt mit guten Einführungen zu dem Thema
(siehe Ende dieses Vorwortes).
Mit einem Schmunzeln könnte man sagen,
dass das Alleinstellungsmerkmal dieses Skriptes seine
\emph{Unvollständigkeit} ist.
Im Gegensatz zu anderen Einführungen wollen wir das Fach nicht in der Breite darstellen,
sondern uns einen Gedankengang, eine spezifische Überlegung,
ein \emph{Narrativ} herauspicken:
Wir wollen die Frage beantworten,
warum diejenigen
schneller, effizienter und präziser
in ihrer täglichen Arbeit als IT-Fachkräfte sind,
die das Kompendium der theoretischen Informatik beherrschen.

Diesem Gedanken folgend werden wir in Kapitel 1 zunächst genauer analysieren,
was überhaupt ein Problem im Sinne der theoretischen Informatik ist
und wie wir es mit Hilfe von Abstraktion und Formalisierung mit den Werkzeugen
dieses Faches bearbeiten können.

TODO: Weitere Kapitelaufbau entlang des Erzählstrangs.

TODO: Lineares Lesen.

Zurück zur Daseinsberechtigung dieses Skriptes:
Es komplementiert andere Einführungen,
viele zentrale Ergebnisse, Hintergründe und Konzepte
der theoretischen Informatik,
die für den gerade dargestellten Erzählstrang irrelvant sind,
finden sich nur in Fußnoten und Referenzen:
Wir werden nichts über Abzählbarkeit,
wenig über Abschlusseigenschaft formaler Sprachen,
linear-beschränkte Turingmaschinen,
oder Transduktoren zu sagen haben.
Aber nach der Lektüre wird hoffentlich klar,
warum eine Vertiefung des Faches lohnenswert ist,
auch in Bezug auf diese ausgesparten Konzepte.

Ein zweites Alleinstellungsmerkmal findet sich im Fokus auf grundlegende
wissenschaftsphilosophische und metatheoretische Fragstellungen,
die ansonsten in anderen Einführungen eher am Rande gestreift
oder einfach ignoriert werden.
Was genau meinen wir mit Begriffen wie ``Problem'', ``Berechnung'', ``Effizienz''?
Welche Rolle spielen Konventionen in der theoretischen Informatik?
Welche Alternativen gibt es zum Methodenkoffer der theoretischen Informatik
bei praktischen Fragestellungen
und warum haben letztere trotzdem mehr als eine Daseinsberechtigung?
Diese Fragen können leicht abgetan werden, frei nach dem Fregeschen Dictum
``philosophia sunt, non leguntur!''\footnote{
    In etwa: ``Das ist Philosophie,
    das lesen wir (als Informatiker:innen) nicht!''
    Frege beklagt in \cite{frege_grundgesetze} (7),
    dass der Mehrwert seiner Beiträge weder in der akademischen Philosophie,
    noch in der akademischen Mathematik Anerkennung finden.
}
Das Buch verfolgt aber den Ansatz,
dass sich Studierende der theoretischen Informatik
\emph {leichter} mit dem klassischen Handwerkszeug tun,
wenn sie Antworten auf diese philosophischen Fragestellungen mitdiskutieren.
Die kritische Perspektive der (formal-analytischen) Philosophie
verspricht uns mehr Präzision und Klarheit.

Dritter Aspekt: Mehr Schlaglicht auf Komplexitätstheorie.

Das ursprüngliche Skript wurde begleitend zur Veranstaltung
\emph{Theoretische Informatik (Wirtschaftsinformatik)}
an der DHBW Heilbronn erstellt.
Es richtet sich vor allem an Fachhochschul- und BA-Student:innen,
die Informatik eher als Teil- oder Nebenfach belegen.
Desweiteren wird das Buch aus oben genannten Gründen auch für Studierende der Philosophie
in der analytisch-angelsächischen Tradition (d.h. mit einem Fokus auf formales Arbeiten)
eine gewinnbringende Einführung in das Fach bieten.
Für Studierende der Informatik auf Universitätsniveau kann das Skript 
\emph{ergänzend} hilfreich sein,
sollte allerdings mit einem vollständigen Kompendium ergänzt werden.
Die folgende Aufstellung von Einführungen in die theoretische Informatik
bietet geeignete Kandidaten für eine Komplementierung, 
spiegelt den Geschmack des Autors wieder
und hat keinen Anspruch auf Vollständigkeit:
\begin{itemize}
    \item  \cite{hoffmann} bietet ein sehr umfassendes Kompendium.
    \item  Einen kurzen und konzisen Überblick bietet \cite{schoening},
            die Reihenfolge der Kapitel entspricht dem kanonischen Aufbau
            einer universitären Vorlesung.
    \item  Für eine Compiler-zentrierte Einführung ist \cite{hedtstueck}
           zu empfehlen.
    \item  Wer Wert auf vollständige formale Beweise legt,
        wird bei \cite{erkpriese} fündig.
    \item  \cite{neubert} hat die niedrigeste Einstiegshürde (es wird viel und ausführlich erklärt) und hat einen starken Fokus auf Algorithmen bzw. Algorithmik.
    \item  Der (englisch-sprachige) Klassiker ist \cite{hopcroftullman},
        für puristische und historisch interessierte Leser:innen zu empfehlen.
    \item  \cite{barak} ist ein Skript in der Entstehung, es ist englisch und offen im Netz zugänglich.
\end{itemize}



